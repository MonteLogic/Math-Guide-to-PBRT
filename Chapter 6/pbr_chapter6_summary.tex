\documentclass[11pt, a4paper]{article}

% --- Packages ---
\usepackage[utf8]{inputenc}
\usepackage{geometry}
\usepackage{amsmath}
\usepackage{amssymb}
\usepackage{graphicx}
\usepackage{hyperref}
\usepackage{listings}
\usepackage{xcolor}
\usepackage{tcolorbox}

% --- Page Formatting ---
\geometry{top=2.5cm, bottom=2.5cm, left=2.5cm, right=2.5cm}
\setlength{\parindent}{0pt}
\setlength{\parskip}{1em}

% --- Code Snippet Styling ---
\definecolor{codegreen}{rgb}{0,0.6,0}
\definecolor{codegray}{rgb}{0.5,0.5,0.5}
\definecolor{codepurple}{rgb}{0.58,0,0.82}
\definecolor{backcolour}{rgb}{0.95,0.95,0.92}

\lstdefinestyle{mystyle}{
    backgroundcolor=\color{backcolour},   
    commentstyle=\color{codegreen},
    keywordstyle=\color{magenta},
    numberstyle=\tiny\color{codegray},
    stringstyle=\color{codepurple},
    basicstyle=\ttfamily\footnotesize,
    breakatwhitespace=false,         
    breaklines=true,                 
    captionpos=b,                    
    keepspaces=true,                 
    numbers=left,                    
    numbersep=5pt,                  
    showspaces=false,                
    showstringspaces=false,
    showtabs=false,                  
    tabsize=4,
    language=C++
}

\lstset{style=mystyle}

% --- Title ---
\title{\textbf{Physically Based Rendering (4th Ed): Chapter 6 Summary}\\
\large \textit{Shapes: Intersection and Differential Geometry}}
\author{Gemini (For a Developer Audience)}
\date{\today}

\begin{document}

\maketitle

\section*{Introduction: Hitting Things}

Chapter 6 is the ``Collision Detection'' chapter.
We have a Ray: $R(t) = O + t\mathbf{d}$.
We have a Shape (Sphere, Triangle, Cylinder).
We need to find $t$ such that the ray touches the shape.

But PBRT goes deeper. It's not enough to just hit the shape. We need to know \textit{everything} about the surface at that hit point: the normal, the UV coordinates, the tangent vectors, etc. This is called \textbf{Differential Geometry}.

% -------------------------------------------------------------------

\section{The Basic Intersection Equation}

\subsection{The Sphere}
A sphere at the origin with radius $r$ is defined by:
\[ x^2 + y^2 + z^2 - r^2 = 0 \]

Substitute the Ray equation ($x = O_x + t d_x$, etc.) into the Sphere equation:
\[ (O_x + t d_x)^2 + (O_y + t d_y)^2 + (O_z + t d_z)^2 - r^2 = 0 \]

This expands into a standard Quadratic Equation:
\begin{equation}
    A t^2 + B t + C = 0
\end{equation}
Where:
\begin{itemize}
    \item $A = \mathbf{d} \cdot \mathbf{d}$ (Length of direction vector squared, usually 1).
    \item $B = 2 (\mathbf{O} \cdot \mathbf{d})$.
    \item $C = \mathbf{O} \cdot \mathbf{O} - r^2$.
\end{itemize}

\textbf{The Solution:}
Use the Quadratic Formula: $t = \frac{-B \pm \sqrt{B^2 - 4AC}}{2A}$.
\begin{itemize}
    \item If $B^2 - 4AC < 0$: No intersection (Ray misses).
    \item If $B^2 - 4AC > 0$: Two intersections (Entry and Exit). Take the smallest positive $t$.
\end{itemize}

\subsection{The Triangle}
Triangles are the most common shape. PBRT uses the \textbf{Möller-Trumbore algorithm}.
It uses Barycentric Coordinates. Any point $P$ inside a triangle $(A, B, C)$ can be written as:
\[ P = (1 - u - v)A + uB + vC \]
We equate the Ray to the Triangle:
\[ O + t\mathbf{d} = (1 - u - v)A + uB + vC \]
This gives us a system of linear equations to solve for $t$, $u$, and $v$. If $u \ge 0$, $v \ge 0$, and $u+v \le 1$, the hit is valid.

% -------------------------------------------------------------------

\section{Differential Geometry: Knowing the Surface}

Once we hit a surface, we populate a `SurfaceInteraction` object. This is crucial for shading.

\subsection{1. The Normal ($\mathbf{n}$)}
The vector perpendicular to the surface.
\begin{itemize}
    \item \textbf{Sphere}: The normal at point $P$ is simply Normalize($P - Center$).
    \item \textbf{Triangle}: The cross product of two edges. $(B-A) \times (C-A)$.
\end{itemize}
The normal tells us which way the surface is facing, which determines how light reflects.

\subsection{2. Parametric Coordinates ($u, v$)}
Every surface in PBRT is parameterized by $(u, v)$ coordinates.
\begin{itemize}
    \item Used for \textbf{Texture Mapping}.
    \item On a sphere, $u$ maps to longitude ($\phi$) and $v$ maps to latitude ($\theta$).
\end{itemize}

\subsection{3. Partial Derivatives ($\frac{\partial P}{\partial u}, \frac{\partial P}{\partial v}$)}
This is the calculus part.
\begin{itemize}
    \item $\frac{\partial P}{\partial u}$: If I move slightly in the $u$ direction on the texture, which way does the point $P$ move in 3D space?
    \item $\frac{\partial P}{\partial v}$: Same for $v$.
\end{itemize}
These two vectors are \textbf{Tangent} to the surface.
The Cross Product of these tangents gives us the Normal:
\[ \mathbf{n} = \text{Normalize}\left( \frac{\partial P}{\partial u} \times \frac{\partial P}{\partial v} \right) \]

\begin{tcolorbox}[title=Why do we need Derivatives?, colback=green!5!white, colframe=green!75!black]
Why not just store the Normal?
Because of \textbf{Bump Mapping} and \textbf{Anisotropy}.

If you want to simulate brushed metal (Anisotropy), the reflection depends on the direction of the scratches. The derivatives $\frac{\partial P}{\partial u}$ tell us exactly which way the ``scratches'' (texture coordinates) flow across the 3D surface.
\end{tcolorbox}

% -------------------------------------------------------------------

\section{Managing Rounding Error}

Computers use floating-point numbers (floats). Floats are not real numbers; they have gaps.
When a ray hits a surface at $t$, the calculated point $P = O + t\mathbf{d}$ might be slightly \textit{below} the surface due to rounding error.

\textbf{Self-Intersection (Shadow Acne):}
If you spawn a new shadow ray from $P$, it might immediately hit the surface itself because $P$ is technically ``inside'' the object. The surface shadows itself, creating ugly black speckles.

\textbf{The Solution: Robust Spawn}
PBRT uses a robust method to spawn rays.
Instead of starting exactly at $P$, it offsets the origin slightly along the Normal $\mathbf{n}$.
\[ O_{new} = P + \epsilon \mathbf{n} \]
PBRT calculates this $\epsilon$ (epsilon) dynamically based on the magnitude of the coordinates, ensuring the ray definitely clears the surface boundary.

% -------------------------------------------------------------------

\section*{Summary for the Developer}
\begin{enumerate}
    \item \textbf{Ray-Shape Intersection}: It's just algebra. Sphere = Quadratic Formula. Triangle = Linear System.
    \item \textbf{SurfaceInteraction}: The result of an intersection isn't just a point. It's a data structure containing Position, Normal, UVs, and Tangents (Derivatives).
    \item \textbf{Tangents matter}: You need partial derivatives ($\partial P / \partial u$) to do advanced shading like bump mapping or brushed metal.
    \item \textbf{Float Precision}: Never trust `float`. Always offset your rays slightly to avoid self-intersection artifacts.
\end{enumerate}

\end{document}
