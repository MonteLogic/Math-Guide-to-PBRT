\documentclass[11pt, a4paper]{article}

% --- Packages ---
\usepackage[utf8]{inputenc}
\usepackage{geometry}
\usepackage{amsmath}
\usepackage{amssymb}
\usepackage{graphicx}
\usepackage{hyperref}
\usepackage{listings}
\usepackage{xcolor}
\usepackage{tcolorbox}

% --- Page Formatting ---
\geometry{top=2.5cm, bottom=2.5cm, left=2.5cm, right=2.5cm}
\setlength{\parindent}{0pt}
\setlength{\parskip}{1em}

% --- Code Snippet Styling ---
\definecolor{codegreen}{rgb}{0,0.6,0}
\definecolor{codegray}{rgb}{0.5,0.5,0.5}
\definecolor{codepurple}{rgb}{0.58,0,0.82}
\definecolor{backcolour}{rgb}{0.95,0.95,0.92}

\lstdefinestyle{mystyle}{
    backgroundcolor=\color{backcolour},   
    commentstyle=\color{codegreen},
    keywordstyle=\color{magenta},
    numberstyle=\tiny\color{codegray},
    stringstyle=\color{codepurple},
    basicstyle=\ttfamily\footnotesize,
    breakatwhitespace=false,         
    breaklines=true,                 
    captionpos=b,                    
    keepspaces=true,                 
    numbers=left,                    
    numbersep=5pt,                  
    showspaces=false,                
    showstringspaces=false,
    showtabs=false,                  
    tabsize=4,
    language=C++
}

\lstset{style=mystyle}

% --- Title ---
\title{\textbf{Physically Based Rendering (4th Ed): Chapter 7 Summary}\\
\large \textit{Primitives and Acceleration: The $O(\log N)$ Algorithm}}
\author{Gemini (For a Developer Audience)}
\date{\today}

\begin{document}

\maketitle

\section*{Introduction: The Performance Problem}

A typical production scene has 10 billion triangles.
A typical 4K image has 8 million pixels.
If we shoot 1000 samples per pixel, that's 8 billion rays.

\textbf{Naive Approach:}
Loop through every ray. Inside that, loop through every triangle.
\[ \text{Complexity} = \text{Rays} \times \text{Triangles} \approx 8 \cdot 10^9 \times 10 \cdot 10^9 = 8 \cdot 10^{19} \text{ checks.} \]
This would take years to render a single frame.

Chapter 7 is about \textbf{Acceleration Structures}. We need to turn that linear $O(N)$ search into a logarithmic $O(\log N)$ search.

% -------------------------------------------------------------------

\section{Primitives and Aggregates}

PBRT uses an elegant object-oriented hierarchy.

\subsection{The Primitive Interface}
A `Primitive` is the abstract base class for ``things that can be hit.''
It bridges the gap between pure Geometry (Shapes) and Materials.
\begin{itemize}
    \item \textbf{GeometricPrimitive}: Holds a reference to a Shape (e.g., Triangle) and a Material (e.g., Plastic).
    \item \textbf{Aggregate}: A collection of Primitives that \textit{looks} like a single Primitive.
\end{itemize}

This is the \textbf{Composite Pattern}. You can put a Sphere in a box. You can put that box in a bigger box. The ray doesn't care; it just calls `Intersect()`.

% -------------------------------------------------------------------

\section{Bounding Boxes (AABB)}

The core tool for acceleration is the \textbf{Axis-Aligned Bounding Box (AABB)}.
It is defined by two points: $P_{min}$ and $P_{max}$.

\textbf{The Logic:}
Checking intersection with a Box is extremely fast.
\begin{itemize}
    \item If a ray misses the Box, it definitely misses everything \textit{inside} the Box.
    \item We only check the contents if we hit the Box.
\end{itemize}

\textbf{Ray-Box Intersection (Slab Method):}
A 3D box is just the intersection of 3 pairs of parallel planes (Slabs).
\begin{itemize}
    \item X-Slab: Region between $x_{min}$ and $x_{max}$.
    \item Y-Slab: Region between $y_{min}$ and $y_{max}$.
    \item Z-Slab: Region between $z_{min}$ and $z_{max}$.
\end{itemize}
The ray hits the box if and only if the intervals of intersection with all three slabs overlap.

% -------------------------------------------------------------------

\section{Bounding Volume Hierarchies (BVH)}

The BVH is the industry standard acceleration structure. It is a binary tree.

\subsection{Structure}
\begin{itemize}
    \item \textbf{Root Node}: A big box containing the entire scene.
    \item \textbf{Internal Nodes}: Have 2 children. They divide the primitives into two groups.
    \item \textbf{Leaf Nodes}: Contain actual triangles (usually 1 to 4).
\end{itemize}

\subsection{Traversal (The Algorithm)}
When a ray hits a Node:
\begin{enumerate}
    \item Does it hit the Node's Bounding Box?
    \item \textbf{No}: Return false. (Pruned! We just skipped millions of triangles).
    \item \textbf{Yes}:
    \begin{itemize}
        \item If Leaf: Check intersection with actual triangles.
        \item If Internal: Recursively check Left Child and Right Child.
    \end{itemize}
\end{enumerate}

\textbf{Optimization}: Visit the \textbf{closer} child first. If we find a hit at distance $t=5$ in the close child, and the far child's box starts at distance $t=10$, we can skip the far child entirely.

% -------------------------------------------------------------------

\section{Building the BVH (SAH)}

How do we build a good tree?
\begin{itemize}
    \item \textbf{Bad Split}: Put 1 tiny triangle in Left Child, and 999,999 triangles in Right Child. (Still $O(N)$).
    \item \textbf{Good Split}: Divide primitives so that the cost of traversing both sides is minimized.
\end{itemize}

PBRT uses the \textbf{Surface Area Heuristic (SAH)}.
It estimates the ``cost'' of a split based on probability.
\begin{equation}
    Cost = C_{trav} + P(L) \cdot C(L) + P(R) \cdot C(R)
\end{equation}
\begin{itemize}
    \item $C_{trav}$: Cost to traverse a node (memory fetch).
    \item $P(L)$: Probability that a random ray hits the Left Child box. This is proportional to the \textbf{Surface Area} of the box.
    \item $C(L)$: Cost of the Left Child (number of triangles).
\end{itemize}

\textbf{The Algorithm:}
1. Sort primitives along an axis (e.g., X-axis).
2. Test different split positions (buckets).
3. Calculate SAH Cost for each split.
4. Pick the split with the lowest cost.
5. Recurse.

\begin{tcolorbox}[title=Why Surface Area?, colback=orange!5!white, colframe=orange!75!black]
Geometric Probability tells us that the probability of a random ray intersecting a convex shape inside a larger box is proportional to its \textbf{Surface Area}.
Minimizing the surface area of the child boxes minimizes the chance that we have to process them.
\end{tcolorbox}

% -------------------------------------------------------------------

\section*{Summary for the Developer}
\begin{enumerate}
    \item \textbf{The Goal}: Reduce complexity from Linear $O(N)$ to Logarithmic $O(\log N)$.
    \item \textbf{AABB}: The fundamental unit. Fast to check. If you miss the box, you skip the contents.
    \item \textbf{BVH}: A tree of boxes.
    \item \textbf{SAH (Surface Area Heuristic)}: The math used to decide how to split the tree. It balances ``balanced tree depth'' vs. ``minimizing empty space.''
    \item \textbf{Traversal}: Depth-first search. Visit closer nodes first to find a hit early and prune the rest.
\end{enumerate}

\end{document}
